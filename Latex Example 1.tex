%%%%%%%%%%%%%%%%%%%%%%%%%%%%%%%%%%%%%%%%%%%%%%%%%%%%%%%%%%%%
%%%%%%%%%%%%%%%%%%%%%%%%%%%%%%%%%%%%%%%%%%%%%%%%%%%%%%%%%%%%
%%%%%%%%%%%%%%%%%%%%%%%%%%%%%%%%%%%%%%%%%%%%%%%%%%%%%%%%%%%%
%%%%%%%%%%%%%%%%%%%%%%%%%%%%%%%%%%%%%%%%%%%%%%%%%%%%%%%%%%%%
%%%%%%%%%%%%%%%%%%%%%%%%%%%%%%%%%%%%%%%%%%%%%%%%%%%%%%%%%%%%
\documentclass[12pt]{article}
\usepackage{epsfig}
\usepackage{times}
\usepackage{amsmath}
\renewcommand{\topfraction}{1.0}
\renewcommand{\bottomfraction}{1.0}
\renewcommand{\textfraction}{0.0}
\setlength {\textwidth}{6.6in}
\setlength{\parindent}{4em}
\hoffset=-1.0in
\oddsidemargin=1.00in
\marginparsep=0.0in
\marginparwidth=0.0in                                                                               
\setlength {\textheight}{9.0in}
\voffset=-1.00in
\topmargin=1.0in
\headheight=0.0in
\headsep=0.00in
\footskip=0.50in                                         
\setcounter{page}{1}
\begin{document}
\def\pos{\medskip\quad}
\def\subpos{\smallskip \qquad}
\newfont{\nice}{cmr12 scaled 1250}
\newfont{\name}{cmr12 scaled 1080}
\newfont{\swell}{cmbx12 scaled 800}
%%%%%%%%%%%%%%%%%%%%%%%%%%%%%%%%%%%%%%%%%%%%%%%%%%%%%%%%%%%%%%%%%
%%%%%%%%%%%%%%%%%%%%%%%%%%%%%%%%%%%%%%%%%%%%%%%%%%%%%%%%%%%%%%%%%
\begin{center}
{\large
PHYS 20323/60323: Fall 2020 - LaTeX Example}\\
%%%%%%%%%%%%%%%%%%%%%%%%%%%%%%%%%%%%%%%%%%%%%%%%%%%%%%%%%%%%%%%%%
\end{center}
\begin{description} 
\item 1. Consider a particle confined in a two-dimensional infinite square well  %
\begin{equation*}
V(x,y)=
\begin{cases}
	0,\;  if 0 \leq x \leq a, \; 0<y<a\\ 	
	\infty,\;  otherwise 
\end{cases}
\end{equation*}
\end{description}
%%%%%%%%%%%%%%%%%%%%%%%%%%%%%%%%%%%%%%%%%%%%%%%%%%%%%%%%%%%%%%%%%
\begin{description}
\item The eigenfunctions have the form: %
\end{description}
\begin{equation*}
\Psi(x,y)=
\frac{2}{a} \sin \left(\frac{n\pi x}{a}\right) \sin\left(\frac{m\pi y}{a}\right)
\end{equation*}
%%%%%%%%%%%%%%%%%%%%%%%%%
\begin{description}
\item with the corresponding energies being given by: %
\end{description}
%%%%%%%%%%%%%%%%%%%%%%%%%%%%%
\begin{equation*}
E_{nm} =
\left(n^2 + m^2\right) \frac{\pi^2 h^2}{2ma^2}
\end{equation*}	
%%%%%%%%%%%%%%%%%%%%%%%%%%%%%%%%%%%%%%%%%%%%%%%%%%%%%%%%%
\begin{description}
\item(a) (5 points) What are the levels of degeneracy of the five lowest energy values? 
\item(b) (5 points) Consider a pertubation given by: 
\end{description}
\begin{equation*}
{H}' =
a^2V_0\delta \left(x-\frac{a}{2}\right)\delta \left(y- \frac{a}{2}\right)
\end{equation*}	
%%%%%%%%%%%%%%%%%%%%%%%%%%%%
\begin{description}
\item Calculate the first order correction to the ground state energy.	
\end{description}
%%%%%%%%%%%%%%%%%%%%%%%%%%%%%%%%%%%%%%%%%%%
\begin{description}
\item \textbf{2. The following questions refer to stars in the Table below.}
\item{Note: There may be multiple answers.}
\end{description}
%%%%%%%%%%%%%%%%%%%%%%%%%%%%%%%%%%%%
\begin{tabular}{|c|c|c|c|c|c|}\hline\hline
Name & Mass & Luminosity & Lifetime & Temperature & Radius\\\hline
Zeta & $60. M_{sun}$ & $10^6 L_{sun}$ & $8.0 \times 10^5$ years & & \\\hline
Epsilon & $6.0 M_{sun}$ & $10^3 L_{sun}$ & & $20,000 K $ & \\\hline
Delta & $2.0 M_{sun}$ & & $5.0 \times 10 ^8$ years & & $2 R_{sun}$ \\\hline
Beta & $1.3 M_{sun}$ & $3.5 L_{sun}$ & & & \\\hline
Alpha & $1.0 M_{sun}$ & & & & $1 R_{sun}$ \\\hline
Gamma & $0.7 M_{sun}$ & & $4.5 \times 10^10$ years & $5000 K $ &\\\hline
\end{tabular}
%%%%%%%%%%%%%%%%%%%%%%%%%%%%%%%%%%%%%%%%%%%%%%%
\begin{description}
\item (a) (4 points) Which of these stars will produce planetary nebula at the end of their life.
\vskip0.2in
\item (b) (4 points) Elements heavier than \textit{Carbon} will be produced in which stars. 
\end{description}
%%%%%%%%%%%%%%%%%%%%%%%%%%%%%%%%%%%%%%%%%%%


\end{document}





